\documentclass[12p]{article}
\usepackage[margin=1in, headheight=110pt]{geometry}
\usepackage{amssymb, amsmath, amsfonts, amsthm}
\usepackage{mathpazo}
\usepackage{setspace}
% \usepackage{probsoln}
\usepackage{fancyhdr}
\usepackage{hyperref}
\usepackage{float}
\usepackage{tikz}
\usepackage{enumitem}
\usepackage{listings}
\usepackage{caption}
\usepackage{bookmark}

\captionsetup{labelformat=empty}
% \usepackage{parskip} % Use for extra line spacing
\def\ojoin{\setbox0=\hbox{$\bowtie$}%
  \rule[-.02ex]{.25em}{.4pt}\llap{\rule[\ht0]{.25em}{.4pt}}}
\def\leftouterjoin{\mathbin{\ojoin\mkern-5.8mu\bowtie}}
\def\rightouterjoin{\mathbin{\bowtie\mkern-5.8mu\ojoin}}
\def\fullouterjoin{\mathbin{\ojoin\mkern-5.8mu\bowtie\mkern-5.8mu\ojoin}}

\pagestyle{fancy}
\lhead{Evan Quan 10154242 - Eddy Qiang 30058191}
\rhead{CPSC 453 - Assignment 3 - T03 - Fall 2018}
\title{Curves \& Splines}
\date{\vspace{-12ex}}

\newcommand{\sep}{\;}
\newtheorem{theorem}{Theorem}
\newtheorem{lemma}[theorem]{Lemma}
\newtheorem{prop}[theorem]{Proposition}
\newtheorem{cor}[theorem]{Corollary}
\newtheorem{corollary}[theorem]{Corollary}
\theoremstyle{definition}
\newtheorem{definition}[theorem]{Definition}
\newcounter{problem}
\newcounter{exercise}
\newcounter{solution}
\newcounter{question}

\newcommand{\question}[3]{%
\textbf{Question #1}: #2

\textbf{Answer:} #3
\vspace{6pt}
}

\setlength{\parindent}{0pt}
\begin{document}
\maketitle
\thispagestyle{fancy}

\onehalfspacing

\begin{center}
  October 15, 2018
\end{center}

\section{B\'ezier Curves}
\begin{enumerate}[label=\Alph*)]
  \item
  \begin{align*}
    B(u) &= (1 - u)^2 \textbf{p}_0 + 2 (1 - u) \textbf{p}_1 + u^2 \textbf{p}_2 \\
         &= (1 - 0.5)^2 \textbf{p}_0 + 2 (0.5)(1 - 0.5) \textbf{p}_1 + (0.5)^2 \textbf{p}_2 \\
         &= \frac{1}{4} \textbf{p}_0 + \frac{1}{2} \textbf{p}_1 + \frac{1}{4} \textbf{p}_2 \\
  \end{align*}
  \begin{align*}
    \textbf{p}_2 &= (1, 0), \textbf{p}_1 = (1, 1), \textbf{p}_2 = (0, 1) \\
    x &= \frac{1}{4} (1) + \frac{1}{2} (1) + \frac{1}{4} (0) = 0.75 \\
    y &= \frac{1}{4} (0) + \frac{1}{2} (1) + \frac{1}{4} (1) = 0.75
  \end{align*}
  \item The shape is a squircle, which approximates a circle. It is not quite a
  circle since the radius is not constant.
  \item
  %% NOTE: 2nd attempt
  The 4 control points are $\textbf{p}_0$, $\textbf{p}_1$, $\textbf{p}_2$ and $\textbf{p}_3$. Since we are
  given the end points we know that $\textbf{p}_0 = (0, 1)$ and $\textbf{p}_1 = (1, 0)$. We are
  also given the tangent of those points, so we know $\textbf{p}_1 = (x_1, 1)$ and $\textbf{p}_2 =
  (1, y_2)$. Since this is a cubic B\'ezier and we are given the midpoint
  $(\frac{\sqrt{2}}{2}, \frac{\sqrt{2}}{2})$, we know that $\frac{\sqrt{2}}{2}$
  is the midpoint of of a midpoint of a midpoint. We can solve for $x_1$ albegraically:

  \begin{align*}
    \frac{\sqrt{2}}{2} = \frac{\Bigg(\bigg(\frac{\frac{0+x_1}{2} + \frac{x_1 + 1}{2}}{2}\bigg) + \bigg(\frac{\frac{1+1}{2} + \frac{1 + x_1}{2}}{2}\bigg)\Bigg)}{2}
    % 2
    % \frac{\sqrt{2}}{2}
  \end{align*}
  Solving for $x_1$, we get $x_1 = \frac{4\sqrt{2}}{3} - \frac{4}{3} = 0.55228$. Since this curve is symmetric about $y = x$, the reflected y value will be the same, therefore $y_2 = 0.55228$. Answer: the control points are (0, 1), (0.55228, 1), (01, 0.55228), (1, 0).



  \item No, the curve does not perfectly form part of a circular arc, only approximate a circular arc. The equation is not exactly
  equivalent to that of a circle as the radius is not constant.

  \begin{align*}
    u = \frac{2}{3}\\
    p_x(\frac{2}{3}) &= (1 - \frac{2}{3})^3 p_{0,x} + 3(\frac{2}{3})(1 - \frac{2}{3})^2 p_{1,x} + 3 (\frac{2}{3})^2(1 - \frac{2}{3}) p_{2, x} + (\frac{2}{3})^3 p_{3, x} \\
    &\approx 0.027(0) + 0.1889 (0.55223) + 0.441(1) + 0.343 (1)\\
    % &= 0.8884\\
    &= 0.863469 \dots \\
    p_y(\frac{2}{3}) &= (1 - \frac{2}{3})^3 p_{0,y} + 3(\frac{2}{3})(1 - \frac{2}{3})^2 p_{1,y} + 3 (\frac{2}{3})^2(1 - \frac{2}{3}) p_{2, y} + (\frac{2}{3})^3 p_{3, y} \\
    &\approx 0.027(1) + 0.1889 (1) + 0.441(0.55223) + 0.343 (0)\\
    % &= 0.4596\\
    &= 0.504717 \dots \\
  \end{align*}
  To calculate the radius:
  \begin{align*}
    r &= \sqrt{x^2 + y^2} \\
    % &= \sqrt{0.8884^2 + 0.4596^2} \\
    % &= 1.00021334
    &= \sqrt{0.863469^2 + 0.504717^2} \\
    &= 1.00015953 \\
    & \neq 1
  \end{align*}
  This is close to, but not exactly 1, meaning that the arc is not that of a perfect
  circle.
  % To prove this, we can acquire the values at:
  % TODO prove with theta = 0, 45, 30
  \item
    The four control points of the of the cubic B\'ezier that form a curve
    identical to that in 1A can be found as it is close to that of 1C. However,
    the midpoint is now $(\frac{3}{4}, \frac{3}{4})$. We can safely conclude
    that the midpoints maintain the same form:
    \begin{align*}
      p_0 &= (0, 1) \\
      p_1 &= (c, 1) \\
      p_2 &= (1, c) \\
      p_3 &= (1, 0)
    \end{align*}
    To find $c$:
    \begin{align*}
      \frac{3}{4} &= p_x (0.5) = (1 - 0.5)^3 p_{0,x} + 3(0.5)(1 - 0.5)^2 p_{1,x} + 3(0.5)^2(1 - 0.5)*p_{2, x} + (0.5)^3 p_{3,x}\\
      \frac{3}{4} &= \frac{1}{8}(0) + \frac{3}{8}c + \frac{3}{8}(1) + \frac{1}{8}(1)\\
      \frac{3}{4} &= \frac{4}{8} + \frac{3}{8}c \\
      6 &= \frac{4}{8} + \frac{3}{8}c \\
      c = \frac{2}{3}
    \end{align*}
    So therefore, our points are:
    \begin{itemize}
      \item $\textbf{p}_0 = (0, 1)$
      \item $\textbf{p}_1 = (\frac{2}{3}, 1)$
      \item $\textbf{p}_2 = (1, \frac{2}{3})$
      \item $\textbf{p}_3 = (1, 0)$
    \end{itemize}
\end{enumerate}

\section{Rendering Fonts}
\begin{enumerate}[label=\Alph*)]
  \item All the knots are $C^0$ continuous since the shape as a whole is
  continuous. In other words, there are no gaps between adjacent knots.
  \item 4, 11, 2, 3, 5, 6, 9, 10, 12, 13 are $G_1$ continuous because the
  tangents are collinear. (The corner knots 1, 7, 8, and 14 are not $G_1$
  continuous since the tangents are not collinear).
  \item 5, 10 are $C_1$ continuous since they have collinear tangents, as needed
  to be $G_1$ continuous, and the adjacent off-curve control points are of
  equivalent magnitude.
\end{enumerate}

\section{Scrolling Fonts}
\begin{enumerate}[label=\Alph*)]
  \item If a 1000 px wide window can fit 10 letters, then a letter is 100 px
  wide. So to move 3 pixels every frame, we would need to 300 px every frame. So
  we need 300 px/s at 60 fps. $\frac{300 px}{60 \text{ frame}} = 5
  px/\text{frame}$. Therefore we would have to move 5 pixels per frame.
  \item $\frac{300 px}{24\text{frame}} = 12.5 px/\text{frame}$. We would have to
  move 12.5 pixels per second. Since we can only move whole integer values of
  pixels, one possible solution to this problem would be to alternate between
  moving 12 and 13 pixels per frame to keep the average of 12.5 pixels per
  second.
\end{enumerate}

% 1c
% d = 4/3(sqrt(2) - 1)
%
\end{document}
